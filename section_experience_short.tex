% Awesome CV LaTeX Template
%
% This template has been downloaded from:
% https://github.com/huajh/huajh-awesome-latex-cv
%
% Author:
% Junhao Hua


%Section: Work Experience at the top
\sectionTitle{项目经历}{\faCode}
 
\begin{experiences}
	
	\experience
	{至今} {BYTON汽车}{车联网研发}{Gateway/TBOX开发}
	{2018年12月 }    {
		\begin{itemize}
			\item 车联网功能定义:根据车辆硬件能力评估整车功能可行性,如空中升级,远程车控,无感启动等
			\item 需求分析,根据整车功能需求制定车联网与FIC/BCM/DCM/ADAS等模块间接口,包括但不限于CAN Matrix,DBC,UDS
			\item 中央网关功能实现,负责CAN message/signal logging in MDF format, UDS stack等功能的编码测试,包括车端的部分C++代码
			\item 网页端功能实现,负责用JavaScript/React实现根据dbc解析CAN报文的前端功能
			\item 基于BLE技术的车辆进入系统功能实现,负责CC2642平台上基于TI SimpleLink SDK的部分编码测试,e.g 片外存储升级功能,启动时低电压保护功能,用于承载车控指令的characteristic,erase/re-pair流程,discovery流程等
			\item Xilinx BSP集成和配置
			\item 基于Buildroot的工具链配置和集成,负责将项目需要的依赖项集成进Buildroot defconfig,例如uboot, kernal patch等
			\item 基于swupdate和Uboot的A/B分区升级功能实现,负责配置环境参数和升级脚本
			\item 为产线测试定制测试逻辑和固件
			\item Android端测试工具实现:负责开发手机端的App,实现被动扫描/日志传送/白名单修改等功能,提高日常开发工作效率
			\item Android手机端无感进入功能测试:负责评审和测试Android手机端到端无感进入功能
			\item 线束定制:负责定制测试用线束,节约项目周期和开销
			\item BQE、EMC认证技术支持:负责为认证团队提供TI芯片相关的认证方案,定制操作步骤
		\end{itemize}
	}
	{嵌入式C、C++,BSP on MPSoC,安卓,前端,CAN,LIN,UDS,DBC,MDF4,BLE}
	
	\emptySeparator	
	\experience
	{2018年12月}{瞬联软件}{爱立信离岸软件咨询项目组}{IoT接入网层2开发}
	{2015年6月} {
		\begin{itemize}
            \item 带领两个XFT小组,实现新feature并维护track质量
			\item 基于3GPP TS 45.008的定义实现eDRX功能
			\item 监控持续集成状态,及时干预新增issue
		\end{itemize}
	}
	{团队建设, 持续集成}
		
	\emptySeparator
	\experience
	{2015年6月} {Lucent Tech}{核心网研发}{后台数据库开发,网管平台开发部署}
	{2004年5月}    {
		\begin{itemize}
			\item 核心网节点间数据库同步机制功能实现
			\item 核心网和接入网更细粒度的性能指标功能实现
			\item 基于CORBA的网管平台的部署,大中华区驻场技术支持
		\end{itemize}
	}
	{异构架构,CMMI,NMS,软件交付流程,文档编写规范} 
	\emptySeparator
\end{experiences}
