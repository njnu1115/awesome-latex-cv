% Awesome CV LaTeX Template
%
% This template has been downloaded from:
% https://github.com/huajh/huajh-awesome-latex-cv
%
% Author:
% Junhao Hua


%Section: Work Experience at the top
\sectionTitle{Experience}{\faCode}
 
\begin{experiences}
	
	\experience
	{Now} {Syton Autotech}{Telematics and Gateway}{Product Mgmt}
	{2018DEC}    {
		\begin{itemize}
			\item Lead the design and delivery of gateway features including but not limited to CAN logging, CAN/Eth routing, diagnostic
			\item Lead the design of CAN log parser front end
			\item Lead the design of dual partition upgrade functionality, based on u-boot and swupdate
			\item Lead the requirement definition of Telematics, and interface definition between telematics and FIC/BCM/DCM/ADAS,including but not limited to CAN Matrix,DBC,UDS
			\item Lead the design of keyless entry features, including off-chip OTA upgrade, low-voltage detection, BLE characteristic for vehicle 
			\item Support Xilinx BSP configuration and integration
			\item Support cross-compilation toolchain configuration and integration
control commands,erase/re-pair, discovery. using TI SimpleLink CC26x2 SDK
			\item Support customize verification procedure and firmware for manufacturing
			\item Support design of Android App for passive BLE scan, log transferring, whitelist modification, etc
		\end{itemize}
	}
	{Product Mgmt, Risk Control, BSP, toolchain}
	\emptySeparator	
	\experience
	{2018DEC}{Cienet Software}{IoT Accessing Network Consulting}{Head of Software Services}
	{2015JUN} {
		\begin{itemize}
			\item Managing and leading 2 XFT teams for new IoT features and track maintenance
			\item Lead the design of eDRX feature defined in 3GPP TS 45.008, embedded C developing on PowerPC platform, targeting reducing the power consumer of Raido Base Station.
			\item Lead the design of EC-GSM feature defined in 3GPP TS 45.008, embedded C developing on PowerPC platform, targeting extending cellular network coverage without hardware investment.
			\item Lead the design of Continuous Integration workflow, including but not limited to equipment interface abstraction(by python), resource allocation and configuraiton framework(by Java), jenkins, etc
		\end{itemize}
	}
	{Team coaching, CI}
	\emptySeparator
	\experience
	{2015JUN} {Lucent Tech}{Core Network}{Developer}
	{2004MAY}    {
		\begin{itemize}
			\item Implement sync mechanism between heterogeneous nodes of core network,
			\item granularity improvement on performance count features
			\item deployment of network management system
		\end{itemize}
	}
	{Heterogeneous architecture, CMMI, NMS}
\end{experiences}
