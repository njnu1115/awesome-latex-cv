% Awesome CV LaTeX Template
%
% This template has been downloaded from:
% https://github.com/huajh/huajh-awesome-latex-cv
%
% Author:
% Junhao Hua


%Section: Work Experience at the top
\sectionTitle{Experience}{\faCode}
 
\begin{experiences}
	
	\experience
	{Now} {BYTON}{Telematics and Gateway}{Developer}
	{2018DEC}    {
		\begin{itemize}
			\item Design of features like CAN logging  UDS stack on Vehicle Gateway, using C/C++
			\item Design of CAN log parser, using JavaScript
			\item Xilinx BSP configuration and integration
			\item Cross-compilation toolchain configuration and integration
			\item Design of dual partition upgrade functionality, based on u-boot and swupdate
			\item Telematics requirements analyze, define interface between telematics and FIC/BCM/DCM/ADAS,including but not limited to CAN Matrix,DBC,UDS
			\item Lead the design of keyless entry features, including off-chip OTA upgrade, low-voltage detection, BLE characteristic for vehicle control commands,erase/re-pair, discovery. using TI SimpleLink CC26x2 SDK
			\item Customize verification procedure and firmware for manufacturing
			\item Implement Android App for passive BLE scan, log transferring, whitelist modification, etc
		\end{itemize}
	}
	{Embedded C/C++, BSP on MPSoC, toolchain, JavaScript, CAN,LIN,UDS,DBC,MDF,Android,BLE,Telematics}
	
	\emptySeparator	
	\experience
	{2018DEC}{Cienet Software}{IoT Accessing Network Consulting}{Head of Software Services}
	{2015JUN} {
		\begin{itemize}
            \item Managing and leading 2 XFT teams for new IoT features and track maintenance
			\item Implement eDRX feature defined in 3GPP TS 45.008, embedded C developing on PowerPC platform, targeting reducing the power consumer of Raido Base Station.
			\item Implement EC-GSM feature defined in 3GPP TS 45.008, embedded C developing on PowerPC platform, targeting extending cellular network coverage without hardware investment.
			\item Monitoring the Continuous Integration health status and do first aid, including but not limited to equipment interface abstraction(by python), resource allocation and configuraiton framework(by Java), jenkins, etc
		\end{itemize}
	}
	{Team coaching, CI}
		
	\emptySeparator
	\experience
	{2015JUN} {Lucent Tech}{Core Network}{Developer}
	{2004MAY}    {
		\begin{itemize}
			\item Implement sync mechanism between heterogeneous nodes of core network, 
			\item granularity improvement on performance count features
			\item deployment of network management system
		\end{itemize}
	}
	{Heterogeneous architecture, CMMI, NMS} 
	\emptySeparator
\end{experiences}
